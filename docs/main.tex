\documentclass[11pt,a4j]{jarticle}
\usepackage{amsfonts}
\usepackage{amsmath}
\usepackage{authblk}
\usepackage[dvipdfmx]{graphicx}
\newcommand\norm[1]{\left\lVert#1\right\rVert}

% タイトル
\title{{KCF Trackerによる高速物体追跡}\thanks{慶應義塾大学 理工学研究科 2019春学期 "ComputerVision特論" 最終レポート} \\ }
% 著者名
\author{内海 佑麻\thanks{情報工学科4年,Email: uchiumi@ailab.ics.keio.ac.jp} (61602452)}
% 日付
\date{2019/07/23}

% 本文
\begin{document}
  % タイトル
  \maketitle
  % 要約
  \begin{abstract}
    モダンな物体追跡手法として,KCF Tracker (Kernelized Correlation Filter Tracker)を紹介し,実装をおこなった.
    KCF Trackerは,データ行列に巡回性を持たせることにより,DCTによる対角化が可能となり,処理にかかるストレージおよび計算量を大幅に削減することができる.
    さらに,カーネルトリックにより

  \end{abstract}
  % 目次
  \tableofcontents
  % 本文
  \section{はじめに}
    \subsection{Optical flow}
      異なる時点の2画像間で,各画素値の移動量を表現したベクトルをOptical Flowという.
      Optical Flowにより特徴点を検出することで,たとえば動画内の特定の物体を追跡することができる.
      まず,一般論として,勾配法(Gradient-based method)によるOptical Flowの求解手順を定式化する.
      勾配法では,Taylor展開による近似を用いるため,「連続する2画像での対象物の移動が微小であること」を前提としている.

      \paragraph{勾配法によるOptical Flowの球解}
      画像$I$における画素$(x,y)$の時刻$t$における画素値を$I(x,y,t)$,時間$\delta t$に対する画素値の移動量を$(\delta x, \delta y)$とおく.
      移動前後で対象画素の画素値が不変であると仮定すると,
      \begin{align}
        I(x,y,t) = I(x + \delta x, y + \delta y, \delta t)
      \end{align}
      が成り立つ.さらに,画素値の変化が滑らかであると仮定すると,1次までのTaylor展開により,以下の近似式を得る.
      \begin{align}
        I(x,y,t) \approx I(x,y,t) + \frac{\partial I}{\partial x} \delta x + \frac{\partial I}{\partial y} \delta y + \frac{\partial I}{\partial t} \delta t
      \end{align}
      両辺を$\delta t$で割って,整理すると,
      \begin{align}
        \frac{\partial I}{\partial x} \frac{\delta x}{\delta t} + \frac{\partial I}{\partial y} \frac{\delta y}{\delta t} + \frac{\partial I}{\partial t} = 0
      \end{align}
      $\delta t \to 0$とすれば,
      \begin{align}
        \frac{\partial I}{\partial x} \frac{\partial x}{\partial t} + \frac{\partial I}{\partial y} \frac{\partial y}{\partial t} + \frac{\partial I}{\partial t} = 0
      \end{align}
      ここで,画素$(x,y)$のOptical Flow: $(u,v) = \left(\frac{\partial x}{\partial t}, \frac{\partial y}{\partial t} \right)$に着目する.
      $I_x = \frac{\partial I}{\partial x}$,$I_y = \frac{\partial I}{\partial y}$,$I_t = \frac{\partial I}{\partial t}$,とおけば,
      $(u,v)$がみたすべき方程式\footnote{これを,OpticalFlowの拘束式と呼ぶ.}は,次式のようになる.
      \begin{align}
        I_x u + I_y v + I_t = 0
      \end{align}
      しかし,上式は2つの未知数$u,v$に対して方程式は1つであり,解が定まらない\footnote{これを,Aperture Problem(窓問題)という.}.
      そのため,上式に加えて,いくつかの仮定(制約条件)を導入してOptical Flow: $(u,v)$を球解する手法が提案されている.
      古典的で重要な手法として,Lucas–Kanade法やHorn–Schunck法がある.
      \begin{itemize}
        \item Lucas–Kanade法 \\
          「各画素の近傍では,移動方向が相関する」という仮定をおく.
          画像中の特定の点(領域)に絞って追跡を行うような用途に適しているため,sparse型と呼ばれる.
        \item Horn–Schunck法 \\
          「各画素は,変化率が最小となる方向へ移動する」という仮定をおく.
          画像中の画素全体の動きを解析するような用途に適しているため,dense型と呼ばれる.
      \end{itemize}


  \section{理論}
    \subsection{Ridge回帰}
      SVMなどの洗練されたモデルに近い性能をもち,かつ単純な閉形式解\footnote{四則演算と初等関数の合成関数によって表せる解を,閉形式解(closed-form solution)という.}をもつRidge回帰を用いる.
      訓練データ${({\bf x}_i, y_i)}_{i = 1 \cdots N}$に対するRidge回帰は,以下のように定式化される.
      \begin{align}
        \max_{\bf w} \sum_{i} { \left( f({\bf x}_i) - y_i \right) }^{2} + \lambda {\norm{ \bf w }}^{2}, ~~~
        f({\bf x}_i) = {\bf w}^{\mathrm{T}} {\bf x}_i
      \end{align}
      Ridge回帰の閉形式解は,以下のようになる.
      \begin{align}
        {\bf w} = { ({X}^{\mathrm{T}} X + \lambda I) }^{-1} {X}^{\mathrm{T}} {\bf y}
      \end{align}
      さらに,転置行列$X^{\mathrm{T}}$を,エルミート転置$X^{\mathrm{H}} := {(X^{*})}^{\mathrm{T}}$によって,拡張すると,
      \begin{align}
        {\bf w} = { ({X}^{\mathrm{T}} X + \lambda I) }^{-1} {X}^{\mathrm{H}} {\bf y}
      \end{align}
      となる.\footnote{$A^{*}$は,行列$A$の複素共役を表す.}

    \subsection{巡回行列とDFT}

      \begin{align}
        P = 
        \left[
          \begin{array}{ccccc}
            0 & 0 & 0 & \cdots & 1 \\
            1 & 0 & 0 & \cdots & 0 \\
            0 & 1 & 0 & \cdots & 0 \\
            \vdots&\vdots&\ddots&\ddots&\vdots \\
            0 & 0 & \cdots & 1 & 1
          \end{array}
        \right]
      \end{align}
      
      巡回行列を定義する.

      \begin{align}
        C({\bf x}) = 
        \left[
          \begin{array}{ccccc}
            x_1 & x_2 & x_3 & \cdots & x_n \\
            x_n & x_1 & x_2 & \cdots & x_{n-1} \\
            x_{n-1} & x_n & x_1 & \cdots & x_{n-2} \\
            \vdots&\vdots&\vdots&\ddots&\vdots \\
            x_2 & x_3 & x_4 & \cdots  & x_1
          \end{array}
        \right]
      \end{align}
    \subsection{カーネルトリック}
  \section{実装}
  \section{考察}
  \section{結論}
    \begin{itemize}
      \item 連続状態空間の制御タスクにおける好奇心駆動の探索戦略: VIMEを提案.
      \item 状態遷移の前後で,KL情報量が最大化するように報酬関数を設計.
      \item モデルパラメータ$\theta$の事後分布を推定する際,変分推論(VI)を用いる.
      \item モデルパラメータ$\theta$は,探索環境のダイナミクスを表現している.
      \item VIMEは,ヒューリスティックを上回る実験結果を示した.
    \end{itemize}

  \bibliography{ref} %hoge.bibから拡張子を外した名前
  \bibliographystyle{junsrt} %参考文献出力スタイル
\end{document}
